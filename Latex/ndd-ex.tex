% This is a LaTeX template kindly taken from Jernej Debevec.
% Provided by Miha Muskinja for the purpose of the seminar I in the 1st year
% of the 2nd cycle of the study of physics at the Faculty of Mathematics and Physics, University of Ljubljana.

% Set the document class and options
\documentclass[10pt, titlepage, a4paper]{article}
\usepackage[a4paper, inner=2.5cm, outer=2.5cm, top=2.25cm, bottom=2.25cm]{geometry}
\usepackage{graphicx}
\usepackage{hyperref}
\usepackage{wrapfig}
\usepackage{amsmath}
\usepackage{amssymb}
\usepackage{bm}
\usepackage{float}
\usepackage{xfrac}
\hypersetup{colorlinks=true}

% Load the natbib package for citation style
\usepackage{natbib}

% Some macros for commonly used symbols in physics/quantum mechanics
\newcommand{\bb}[1]{\bm#1}
\newcommand{\dd}{\mathrm{d}}
\newcommand{\pp}{\partial}
\newcommand{\dg}{\dagger}
\newcommand{\la}{\langle}
\newcommand{\ra}{\rangle}
\newcommand{\id}{\mathbb{1}}
\newcommand{\T}{\mathsf{T}}
\newcommand{\ua}{\uparrow\>}
\newcommand{\da}{\downarrow\>}
\newcommand{\fs}[1]{\slashed{#1}}  % Feynmann slash
\newcommand{\mc}[1]{\mathcal{#1}}
\newcommand\thickbar[1]{\accentset{\rule{.5em}{.03em}}{#1}}
\renewcommand{\bar}{\thickbar}

\numberwithin{equation}{section}

% Start the document
\begin{document}

% The title page
\begin{titlepage}
{\centering
\includegraphics[width=6cm]{logo_fmf.pdf}

\vspace{0.8cm}
{\small Department of Physics}

\vspace{5cm}
\vspace{0.5cm}
{\huge\textbf{Advanced Particle Detectors and Data Analysis}} \\
\vspace{0.5cm}
{\large\textbf{Notes for Exercises}}

\vfill
\textbf{Author:} Marko Urbanč \\
\textbf{Professor:} prof. dr. Peter Križan \\ 
\textbf{Assistant:} doc. dr. Rok Dolenec \\

\vspace{1cm}
Ljubljana, December 2024 \\
}
\vspace{3cm}
\end{titlepage}

% Add table of conents
\hypersetup{pageanchor=true}
\pagenumbering{roman}
\setcounter{page}{2}
\tableofcontents
\vspace{1cm}

% Proceed with the main body
\pagenumbering{arabic}

\section{Interactions of Particles with Photons}
\subsection{Bethe-Bloch Equation}
The Bethe-Bloch equation describes the mean energy loss per distance traveled while traversing through matter.
We generally use the Bethe-Bloch equation when we are dealing with \textbf{thick absorbers}, such as the ones in calorimeters.
Do note that the Bethe-Bloch equation does not accurately describe the energy loss of \textbf{electrons} and \textbf{positrons} 
due to their small mass and the fact that they suffer from much larger energy losses due to bremsstrahlung and pair production.
For a particle with charge $z$ and velocity $\beta = v/c$, the Bethe-Bloch equation is given as:
%
\begin{equation}
    -\left\langle \frac{\dd E}{\dd x} \right\rangle = 2\pi N_a r_e^2 m_e c^2 \rho \frac{Z}{A}\frac{z^2}{\beta^2}\left[\ln\left(\frac{2m_ec^2\beta^2\gamma^2W_{\text{max}}}{I^2}\right) - 2\beta^2 - \delta - 2\frac{C}{Z}\right]\>,
    \label{eq:bethe-bloch}
\end{equation}
%
where $\delta$ is the \textbf{density effect correction} and $C$ is the \textbf{shell correction}. The rest is as follows:
%
\begin{gather*}
    N_a = 6.022 \times 10^{23}\,\text{mol}^{-1}\>, \quad r_e = 2.818 \times 10^{-15}\,\text{m}\>, \quad m_e = 9.11 \times 10^{-31}\,\text{kg}\>, \quad c = 3 \times 10^8\,\text{m/s}\>, \nonumber \\
    \rho = \text{density of the material}\>, \quad A = \text{atomic mass of the material}\>, \quad Z = \text{atomic number of the material}\>, \\
    \quad \gamma = \frac{1}{\sqrt{1-\beta^2}}\>, \quad W_{\text{max}} = \text{maximum energy transfer in a single collision}\>, \quad I = \text{mean excitation energy}\>. \nonumber
\end{gather*}
%
The constant factor in the equation can be written as:
%
\begin{equation}
    \Xi = \bm{2\pi N_a r_e^2 m_e c^2 = 0.1535\,\textbf{MeV cm}^2\textbf{mol}^{-1}}\>,
    \label{eq:bb-constant}
\end{equation}
%
where I've chosen to mark this constant factor as $\Xi$ for easier reference in further calculations.
We can find the mean excitation energy $I$ from the following experimentally determined formula:
%
\begin{equation}
    I = \begin{cases}
        Z(12 + \frac{7}{Z}) & \text{for } Z < 13\>, \\
        Z(9.76 + 58.8Z^{-1.19}) & \text{for } Z \geq 13\>.
    \end{cases}
    \label{eq:I}
\end{equation}
The maximum energy transfer in a single collision $W_{\text{max}}$ can be calculated as:
%
\begin{equation}
    W_{\text{max}} = \frac{2m_e c^2\beta^2\gamma^2}{1 + 2\gamma \left(\sfrac{m_e}{M}\right) + \left(\sfrac{m_e}{M}\right)^2}\approx 2m_ec^2\beta^2\gamma^2\>.
    \label{eq:wmax}
\end{equation}
%
For our purposes we will ignore the density effect correction $\delta$ and the shell correction $C$.

\subsubsection{Energy Loss of Charged Kaons}
Let us calculate the energy losses for charged kaons $K^+$ and $K^-$ with a rest mass of $0.493\,\text{MeV}$ and momentum of $2.5\,\text{GeV}$ in copper which has the
following properties:
%
\begin{align*}
    & \rho = 8.92\,\text{g/cm}^3\>, \\
    & Z = 29\>, \\
    & A = 63.5\,\text{g/mol}\>.
\end{align*}
%
First let us calculate the velocity $\beta$ and the Lorentz factor $\gamma$. We know that
%
\begin{equation}
    \beta = \frac{pc}{E} = \frac{pc}{\sqrt{(pc)^2 + (Mc^2)^2}}\>,
\end{equation}
%
where $M$ is the mass of the particle. Thus:
%
\begin{equation}
    \beta = \frac{2.5\>\dfrac{\rm{GeV}}{c}\>c}{\sqrt{\left(2.5\>\dfrac{\rm{GeV}}{c}\>c\right)^2 + {\left(0.493\>\dfrac{\rm{MeV}}{c^2}\>c^2\right)}^2}} \approx 0.981031\>.
\end{equation}
%
\textbf{Remember to take at least 4 significant digits for the velocity $\beta$!} This is due to the logarithm in the Bethe-Bloch equation.
The Lorentz factor $\gamma$ is then:
%
\begin{equation}
    \gamma = \frac{1}{\sqrt{1-\beta^2}} \approx 5.159\>.
\end{equation}
%
Next let us calculate the maximum energy transfer in a single collision $W_{\text{max}}$:
%
\begin{equation}
    W_{\text{max}} = 2m_ec^2\beta^2\gamma^2 = 2\cdot 0.511\>\frac{\rm{MeV}}{c^2}\>c^2\cdot(0.981031)^2(5.159)^2 = 26.7\>\text{MeV}\>.
\end{equation}
%
Last prerequisite is the mean excitation energy $I$ which we can calculate using the formula (\ref{eq:I}) for $Z \geq 13$:
%
\begin{equation}
    I = 29(9.76 + 58.8\cdot 29^{-1.19})\>\rm{eV} = 313.9\>\rm{eV}\>.
\end{equation}
%
Now all that is left is to plug in the values into the Bethe-Bloch equation (\ref{eq:bethe-bloch}):
%
\begin{flalign}
    -\left\langle \frac{\dd E}{\dd x} \right\rangle &= \Xi\>\rho \frac{Z}{A}\frac{z^2}{\beta^2}\left[\ln\left(\frac{W_{\text{max}}^2}{I^2}\right) - 2\beta^2\right] \\
    &= 0.1535\>\frac{\rm{MeV}\,\rm{cm}^2}{\rm{mol}}\cdot 8.92\>\frac{\rm{g}}{{\rm{cm}}^3}\cdot\frac{29}{63.5}\>\frac{\rm{mol}}{\rm{g}}\cdot\frac{1}{0.981031^2}\cdot\left[\ln\left(\frac{(26.7\cdot10^6\>\rm{eV})^2}{(313.9\>\rm{eV})^2}\right) - 2\cdot (0.981031)^2\right] \nonumber \\
    &= 13.47\>\frac{\rm{MeV}}{\rm{cm}} \>. \nonumber
\end{flalign}

% Add references
% \newpage
% \bibliographystyle{unsrt}
% \bibliography{main}

% End document
\end{document}

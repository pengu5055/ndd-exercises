% This is a LaTeX template kindly taken from Jernej Debevec.
% Provided by Miha Muskinja for the purpose of the seminar I in the 1st year
% of the 2nd cycle of the study of physics at the Faculty of Mathematics and Physics, University of Ljubljana.

% Set the document class and options
\documentclass[10pt, titlepage, a4paper]{article}
\usepackage[a4paper, inner=2.5cm, outer=2.5cm, top=2.25cm, bottom=2.25cm]{geometry}
\usepackage{graphicx}
\usepackage{hyperref}
\usepackage{wrapfig}
\usepackage{amsmath}
\usepackage{amssymb}
\usepackage{bm}
\usepackage{float}
\usepackage{xfrac}
\hypersetup{colorlinks=true}

% Load the natbib package for citation style
\usepackage{natbib}

% Some macros for commonly used symbols in physics/quantum mechanics
\newcommand{\bb}[1]{\bm#1}
\newcommand{\dd}{\mathrm{d}}
\newcommand{\pp}{\partial}
\newcommand{\dg}{\dagger}
\newcommand{\la}{\langle}
\newcommand{\ra}{\rangle}
\newcommand{\id}{\mathbb{1}}
\newcommand{\T}{\mathsf{T}}
\newcommand{\ua}{\uparrow\>}
\newcommand{\da}{\downarrow\>}
\newcommand{\fs}[1]{\slashed{#1}}  % Feynmann slash
\newcommand{\mc}[1]{\mathcal{#1}}
\newcommand\thickbar[1]{\accentset{\rule{.5em}{.03em}}{#1}}
\renewcommand{\bar}{\thickbar}

% Start the document
\begin{document}

% The title page
\begin{titlepage}
{\centering
\includegraphics[width=6cm]{logo_fmf.pdf}

\vspace{0.8cm}
{\small Department of Physics}

\vspace{5cm}
\vspace{0.5cm}
{\huge\textbf{Advanced Particle Detectors and Data Analysis}} \\
\vspace{0.5cm}
{\large\textbf{Notes for Exercises}}

\vfill
\textbf{Author:} Marko Urbanč \\
\textbf{Professor:} prof. dr. Peter Križan \\ 
\textbf{Assistant:} doc. dr. Rok Dolenec \\

\vspace{1cm}
Ljubljana, December 2024 \\
}
\vspace{3cm}
\end{titlepage}

% Add table of conents
\hypersetup{pageanchor=true}
\pagenumbering{roman}
\setcounter{page}{2}
\tableofcontents
\vspace{1cm}

% Proceed with the main body
\pagenumbering{arabic}

\section{Interactions of Particles with Photons}
\subsection{Bethe-Bloch Equation}
The Bethe-Bloch equation describes the mean energy loss per distance traveled while traversing through matter.
We generally use the Bethe-Bloch equation when we are dealing with \textbf{thick absorbers}, such as the ones in calorimeters.
Do note that the Bethe-Bloch equation does not accurately describe the energy loss of \textbf{electrons} and \textbf{positrons} 
due to their small mass and the fact that they suffer from much larger energy losses due to bremsstrahlung and pair production.
For a particle with charge $Z$ and velocity $\beta = v/c$, the Bethe-Bloch equation is given as:
%
\begin{equation}
    -\left\langle \frac{\dd E}{\dd x} \right\rangle = 2\pi N_a r_e^2 m_e c^2 \rho \frac{Z}{A}\frac{z^2}{\beta^2}\left[\ln\left(\frac{2m_ec^2\beta^2\gamma^2W_{\text{max}}}{I^2}\right) - 2\beta^2 - \delta - 2\frac{C}{Z}\right]\>,
    \label{eq:bethe-bloch}
\end{equation}
%
where $\delta$ is the \textbf{density effect correction} and $C$ is the \textbf{shell correction}. The rest is as follows:
%
\begin{gather}
    N_a = 6.022 \times 10^{23}\,\text{mol}^{-1}\>, \quad r_e = 2.818 \times 10^{-15}\,\text{m}\>, \quad m_e = 9.11 \times 10^{-31}\,\text{kg}\>, \quad c = 3 \times 10^8\,\text{m/s}\>, \nonumber \\
    \rho = \text{density of the material}\>, \quad A = \text{atomic mass of the material}\>, \quad z = \text{charge of the particle}\>, \quad \gamma = \frac{1}{\sqrt{1-\beta^2}}\>, \nonumber \\
    W_{\text{max}} = \text{maximum energy transfer in a single collision}\>, \quad I = \text{mean excitation energy}\>.
\end{gather}
%
The constant factor in the equation can be written as:
%
\begin{equation}
    2\pi N_a r_e^2 m_e c^2 = 0.1535\,\text{MeV cm}^2\text{mol}^{-1}\>.
\end{equation}
%
We can find the mean excitation energy $I$ from the following experimentally determined formula:
%
\begin{equation}
    I = \begin{cases}
        Z(12 + \frac{7}{Z}) & \text{for } Z < 13\>, \\
        Z(9.76 + 58.8Z^{-1.19}) & \text{for } Z \geq 13\>.
    \end{cases}
\end{equation}
The maximum energy transfer in a single collision $W_{\text{max}}$ can be calculated as:
%
\begin{equation}
    W_{\text{max}} = \frac{2m_e c^2\beta^2\gamma^2}{1 + 2\gamma \left(\sfrac{m_e}{M}\right) + \left(\sfrac{m_e}{M}\right)^2}\approx 2m_ec^2\beta^2\gamma^2\>.
\end{equation}
%
For our purposes we will ignore the density effect correction $\delta$ and the shell correction $C$.

% Add references
% \newpage
% \bibliographystyle{unsrt}
% \bibliography{main}

% End document
\end{document}

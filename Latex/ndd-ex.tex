% This is a LaTeX template kindly taken from Jernej Debevec.
% Provided by Miha Muskinja for the purpose of the seminar I in the 1st year
% of the 2nd cycle of the study of physics at the Faculty of Mathematics and Physics, University of Ljubljana.

% Set the document class and options
\documentclass[10pt, titlepage, a4paper]{article}
\usepackage[a4paper, inner=2.5cm, outer=2.5cm, top=2.25cm, bottom=2.25cm]{geometry}
\usepackage{graphicx}
\usepackage{hyperref}
\usepackage{wrapfig}
\usepackage{amsmath}
\usepackage{amssymb}
\usepackage{bm}
\usepackage{float}
\usepackage{xfrac}
\usepackage{accents}
\hypersetup{colorlinks=true}

% Load the natbib package for citation style
\usepackage{natbib}

% Set indent to 0
\setlength\parindent{0pt}

% Some macros for commonly used symbols in physics/quantum mechanics
\newcommand{\bb}[1]{\bm#1}
\newcommand{\dd}{\mathrm{d}}
\newcommand{\pp}{\partial}
\newcommand{\dg}{\dagger}
\newcommand{\la}{\langle}
\newcommand{\ra}{\rangle}
\newcommand{\id}{\mathbb{1}}
\newcommand{\T}{\mathsf{T}}
\newcommand{\ua}{\uparrow\>}
\newcommand{\da}{\downarrow\>}
\newcommand{\fs}[1]{\slashed{#1}}  % Feynmann slash
\newcommand{\mc}[1]{\mathcal{#1}}
\newcommand\thickbar[1]{\accentset{\rule{.5em}{.03em}}{#1}}
\renewcommand{\bar}{\thickbar}

\numberwithin{equation}{section}

% Start the document
\begin{document}

% The title page
\begin{titlepage}
{\centering
\includegraphics[width=6cm]{logo_fmf.pdf}

\vspace{0.8cm}
{\small Department of Physics}

\vspace{5cm}
\vspace{0.5cm}
{\huge\textbf{Advanced Particle Detectors and Data Analysis}} \\
\vspace{0.5cm}
{\large\textbf{Notes for Exercises}}

\vfill
\textbf{Author:} Marko Urbanč \\
\textbf{Professor:} prof. dr. Peter Križan \\ 
\textbf{Assistant:} doc. dr. Rok Dolenec \\

\vspace{1cm}
Ljubljana, December 2024 \\
}
\vspace{3cm}
\end{titlepage}

% Add table of conents
\hypersetup{pageanchor=true}
\pagenumbering{roman}
\setcounter{page}{2}
\tableofcontents
\vspace{1cm}

% Proceed with the main body
\pagenumbering{arabic}

\section{Interactions of Particles with Photons}
\subsection{Bethe-Bloch Equation}
The Bethe-Bloch equation describes the mean energy loss per distance traveled while traversing through matter.
We generally use the Bethe-Bloch equation when we are dealing with \textbf{thick absorbers}, such as the ones in calorimeters.
Do note that the Bethe-Bloch equation does not accurately describe the energy loss of \textbf{electrons} and \textbf{positrons} 
due to their small mass and the fact that they suffer from much larger energy losses due to bremsstrahlung and pair production.
For a particle with charge $z$ and velocity $\beta = v/c$, the Bethe-Bloch equation is given as:
%
\begin{equation}
    -\left\langle \frac{\dd E}{\dd x} \right\rangle = 2\pi N_a r_e^2 m_e c^2 \rho \frac{Z}{A}\frac{z^2}{\beta^2}\left[\ln\left(\frac{2m_ec^2\beta^2\gamma^2W_{\text{max}}}{I^2}\right) - 2\beta^2 - \delta - 2\frac{C}{Z}\right]\>,
    \label{eq:bethe-bloch}
\end{equation}
%
where $\delta$ is the \textbf{density effect correction} and $C$ is the \textbf{shell correction}. The rest is as follows:
%
\begin{gather*}
    N_a = 6.022 \times 10^{23}\,\text{mol}^{-1}\>, \quad r_e = 2.818 \times 10^{-15}\,\text{m}\>, \quad m_e = 9.11 \times 10^{-31}\,\text{kg}\>, \quad c = 3 \times 10^8\,\text{m/s}\>, \nonumber \\
    \rho = \text{density of the material}\>, \quad A = \text{atomic mass of the material}\>, \quad Z = \text{atomic number of the material}\>, \\
    \quad \gamma = \frac{1}{\sqrt{1-\beta^2}}\>, \quad W_{\text{max}} = \text{maximum energy transfer in a single collision}\>, \quad I = \text{mean excitation energy}\>. \nonumber
\end{gather*}
%
The constant factor in the equation can be written as:
%
\begin{equation}
    \Xi = \bm{2\pi N_a r_e^2 m_e c^2 = 0.1535\,\textbf{MeV cm}^2\textbf{mol}^{-1}}\>,
    \label{eq:bb-constant}
\end{equation}
%
where I've chosen to mark this constant factor as $\Xi$ for easier reference in further calculations.
We can find the mean excitation energy $I$ from the following experimentally determined formula:
%
\begin{equation}
    I = \begin{cases}
        Z(12 + \frac{7}{Z})\>\rm{eV} & \text{for } Z < 13\>, \\
        Z(9.76 + 58.8Z^{-1.19})\>\rm{eV} & \text{for } Z \geq 13\>.
    \end{cases}
    \label{eq:I}
\end{equation}
The maximum energy transfer in a single collision $W_{\text{max}}$ can be calculated as:
%
\begin{equation}
    W_{\text{max}} = \frac{2m_e c^2\beta^2\gamma^2}{1 + 2\gamma \left(\sfrac{m_e}{M}\right) + \left(\sfrac{m_e}{M}\right)^2}\approx 2m_ec^2\beta^2\gamma^2\>.
    \label{eq:wmax}
\end{equation}
%
For our purposes we will ignore the density effect correction $\delta$ and the shell correction $C$.

\subsubsection{Energy Loss of Charged Kaons}
Let us calculate the energy losses for charged kaons $K^+$ and $K^-$ with a rest mass of $0.493\,\text{GeV}$ and momentum of $2.5\,\text{GeV}$ in copper which has the
following properties:
%
\begin{align*}
    & \rho = 8.92\,\text{g/cm}^3\>, \\
    & Z = 29\>, \\
    & A = 63.5\,\text{g/mol}\>.
\end{align*}
%
First let us calculate the velocity $\beta$ and the Lorentz factor $\gamma$. We know that
%
\begin{equation}
    \beta = \frac{pc}{E} = \frac{pc}{\sqrt{(pc)^2 + (Mc^2)^2}}\>,
    \label{eq:beta}
\end{equation}
%
where $M$ is the mass of the particle. Thus:
%
\begin{equation}
    \beta = \frac{2.5\>\dfrac{\rm{GeV}}{c}\>c}{\sqrt{\left(2.5\>\dfrac{\rm{GeV}}{c}\>c\right)^2 + {\left(0.493\>\dfrac{\rm{GeV}}{c^2}\>c^2\right)}^2}} \approx 0.981031\>.
\end{equation}
%
\textbf{Remember to take at least 4 significant digits for the velocity $\beta$!} This is due to the logarithm in the Bethe-Bloch equation.
The Lorentz factor $\gamma$ is then:
%
\begin{equation}
    \gamma = \frac{1}{\sqrt{1-\beta^2}} \approx 5.159\>.
\end{equation}
%
Next let us calculate the maximum energy transfer in a single collision $W_{\text{max}}$:
%
\begin{equation}
    W_{\text{max}} = 2m_ec^2\beta^2\gamma^2 = 2\cdot 0.511\>\frac{\rm{MeV}}{c^2}\>c^2\cdot(0.981031)^2(5.159)^2 = 26.7\>\text{MeV}\>.
\end{equation}
%
Last prerequisite is the mean excitation energy $I$ which we can calculate using the formula (\ref{eq:I}) for $Z \geq 13$:
%
\begin{equation}
    I = 29(9.76 + 58.8\cdot 29^{-1.19})\>\rm{eV} = 313.9\>\rm{eV}\>.
\end{equation}
%
Now all that is left is to plug in the values into the Bethe-Bloch equation (\ref{eq:bethe-bloch}):
%
\begin{flalign}
    -\left\langle \frac{\dd E}{\dd x} \right\rangle &= \Xi\>\rho \frac{Z}{A}\frac{z^2}{\beta^2}\left[\ln\left(\frac{W_{\text{max}}^2}{I^2}\right) - 2\beta^2\right] \nonumber \\
    &= 0.1535\>\frac{\rm{MeV}\,\rm{cm}^2}{\rm{mol}}\cdot 8.92\>\frac{\rm{g}}{{\rm{cm}}^3}\cdot\frac{29}{63.5}\>\frac{\rm{mol}}{\rm{g}}\cdot\frac{1}{0.981031^2}\cdot\left[\ln\left(\frac{(26.7\cdot10^6\>\rm{eV})^2}{(313.9\>\rm{eV})^2}\right) - 2\cdot (0.981031)^2\right] \nonumber \\
    &= 13.47\>\frac{\rm{MeV}}{\rm{cm}} \>.
    \label{eq:1sub1-result}
\end{flalign}
%
Thus the energy loss of charged kaons $K^+$ and $K^-$ with a momentum of $2.5\,\rm{GeV}$ in copper is $13.47\,\rm{MeV/cm}$.

\subsubsection{What is the Energy Resolution of the Detector from the Previous Example?}
Let's calculate the energy resolution of the detector from the previous example, assuming that the length of the 
particle track through the detector is $d = 5\,\rm{cm}$ and that energy is measured based on all deposited energy without
any additional losses. Using the result from the previous example (\ref{eq:1sub1-result}), we can calculate the average energy deposited 
in the detector as:
%
\begin{equation}
    \Delta E = \bar{\Delta} = -\left\langle \frac{\dd E}{\dd x} \right\rangle \cdot d = 13.47\>\frac{\rm{MeV}}{\rm{cm}}\cdot 5\>\rm{cm} = 67.35\>\rm{MeV}\>.
\end{equation}
%
This is an approximation since we are assuming that $\beta$ is constant throughout the detector, which is not true. In reality we'd 
have to integrate the energy loss over the path of the particle, however at $p\sim\rm{GeV}$ additional losses of $\sim\rm{MeV}$ are 
negligible. Measurements of energy are dependant on the energy resolution $R$ which is defined as:
%
\begin{equation}
    R = \frac{\sigma_E}{\bar{\Delta}}\>,
    \label{eq:energy-res}
\end{equation}
%
where $\sigma_E$ is the standard deviation of the energy measurement which we assume to have a Gaussian distribution like such:
%
\begin{equation}
    p(\Delta) = \frac{1}{\sqrt{2\pi}\sigma_E}\exp\left(-\frac{{\Delta - \bar{\Delta}}^2}{2\sigma_E^2}\right)\>.
\end{equation}
%
$\sigma_E$ is determined empirically. For \textbf{non-relativistic} particles it can be calculated as the variance of the 
Bethe-Bloch equation as:
%
\begin{equation}
    \sigma_0^2 = 4\pi N_a r_e^2 (m_e c^2)^2 \rho \frac{Z}{A}\,\Delta x\>.
    \label{eq:sigE-nonrel}
\end{equation}
%
For \textbf{relativistic} particles we can correct the variance from (\ref{eq:sigE-nonrel}) as such:
%
\begin{equation}
    \sigma_E^2 = \sigma_0^2\>\frac{1-\frac{1}{2}\beta^2}{1-\beta^2}\>.
\end{equation}
%
In our case this gives us:
%
\begin{flalign}
    \sigma_E^2 &= 2\cdot 0.511\>\frac{\rm{MeV}}{c^2}\>c^2\cdot 0.1535\>\frac{\rm{MeV}\rm{cm}^2}{\rm{mol}}\cdot 8.92\>\frac{\rm{g}}{{\rm{cm}}^3}\cdot\frac{29}{63.5}\>\frac{\rm{mol}}{\rm{g}}\cdot 5\>\rm{cm}\cdot\frac{1-\frac{1}{2}(0.981031)^2}{1-(0.981031)^2} \nonumber \\
    &= 44.12\>\rm{MeV}^2\>.
\end{flalign}
%
Thus the energy resolution of the detector is:
%
\begin{equation}
    R = \frac{\sqrt{44.12\>\rm{MeV}^2}}{67.35\>\rm{MeV}} = 9.9\%\>.
    \label{eq:1sub2-result}
\end{equation}
%
\subsubsection{What if the Detector is Made of a Molecule?}
Let's assume now that our detector is made of lead(II) fluoride $\text{PbF}_2$ in a cubic crystal form which has the following properties:
%
\begin{equation*}
    \begin{aligned}
        & \rho = 7.77\>\text{g/cm}^3\>, \\
        & Z = 100\>, \\
        & A = 245.2\>\text{g/mol}\>. \\
        & A_{\rm{Pb}} = 207.2\>\text{u}\>, \\
        & A_{\rm{F} }= 19\>\text{u}\>, \\
    \end{aligned}
    \qquad\qquad
    \begin{aligned}
        & Z_{\rm{Pb}} = 82\>, \\
        & Z_{\rm{F} }= 9\>, \\
        & \rho_{\rm{Pb}} = 11.34\>\text{g/cm}^3\>, \\
        & \rho_{\rm{F}} = 0.001696\>\text{g/cm}^3\>.
    \end{aligned}
\end{equation*}
%
We are interested in the energy loss of protons with a momentum of $3\,\text{GeV}$ in such a detector. The difference between calculating the energy loss in a compound 
material is that we have to calculate the energy loss for each element in the compound. This sum is weighted by the fraction of the element in the compound. As such:
%
\begin{equation}
    \frac{1}{\rho} \left\langle \frac{\dd E}{\dd x} \right\rangle _{\text{compound}} = \frac{w_1}{\rho_1} \left\langle \frac{\dd E}{\dd x} \right\rangle _1 + \frac{w_2}{\rho_2} \left\langle \frac{\dd E}{\dd x} \right\rangle _2 + \ldots\>,
    \label{eq:bb-compound}
\end{equation}
%
where we calculate $w_i$ as:
%
\begin{equation}
    w_i = \frac{a_i\cdot A_i}{\sum a_i\cdot A_i}\>,
\end{equation}
%
here $a_i$ is the number of atoms of the element in the compound and $A_i$ is the atomic mass of the element. Our professor stated that such 
problems will not be present on the exam and that we should not worry about them. However it is still good to know how to calculate the energy loss in a compound.
In our case we can expect to get effective values if the detector is made of a compound. If we calculate the weights for lead and fluorine in lead(II) fluoride we get:
%
\begin{align*}
    w_{\rm{Pb}} = \frac{1\cdot 207.2\>\rm{u}}{1\cdot 207.2\>\rm{u} + 2\cdot 19\>\rm{u}} = 0.845 \>, \\
    w_{\rm{F}} = \frac{2\cdot 19\>\rm{u}}{1\cdot 207.2\>\rm{u} + 2\cdot 19\>\rm{u}} = 0.154 \>,
\end{align*}
%
where we used the atomic masses of lead and fluoride in atomic mass units. Next we need to calculate the velocity $\beta$ and the Lorentz factor $\gamma$ for protons. So using
(\ref{eq:beta}) we get:
%
\begin{equation}
    \beta = \frac{3\>\dfrac{\rm{GeV}}{c}\>c}{\sqrt{\left(3\>\dfrac{\rm{GeV}}{c}\>c\right)^2 + {\left(0.938\>\dfrac{\rm{GeV}}{c^2}\>c^2\right)}^2}} \approx 0.95443\>,
\end{equation}
%
which gives us a Lorentz factor of:
%
\begin{equation}
    \gamma = \frac{1}{\sqrt{1-(0.95443)^2}} \approx 3.351\>.
\end{equation}
%
Next we need to calculate the maximum energy transfer in a single collision $W_{\text{max}}$ using 
(\ref{eq:wmax}) as: 
%
\begin{equation}
    W_{\text{max}} = 2\cdot 0.511\>\frac{\rm{MeV}}{c^2}\>c^2\cdot(0.95443)^2(3.351)^2 = 10.5\>\rm{MeV}\>,
\end{equation}
%
and the mean excitation energy $I$ using (\ref{eq:I}) for each component:
%
\begin{align*}
    I_{\rm{Pb}} &= 82\left(9.76 + 58.8\cdot 82^{-1.19}\right)\rm{eV} = 825.8\>\rm{eV}\>, \\
    I_{\rm{Cu}} &= 9\left(12 + \frac{7}{9}\right)\rm{eV} = 115\>\rm{eV}\>.
\end{align*}
%
Now we can calculate the energy loss for each component using the Bethe-Bloch equation (\ref{eq:bethe-bloch}) and sum them up:
%
\begin{flalign}
    -\left\langle \frac{\dd E}{\dd x} \right\rangle _{\rm{Pb}} &= \Xi\>\rho_{\rm{Pb}} \frac{Z_{\rm{Pb}}}{A_{\rm{Pb}}}\frac{1}{0.95443^2}\left[\ln\left(\frac{(10.5\cdot10^6\>\rm{eV})^2}{(825.8\>\rm{eV})^2}\right) - 2\cdot (0.95443)^2\right] \nonumber \\
    &= \Xi\>\rho_{\rm{Pb}} \frac{Z_{\rm{Pb}}}{A_{\rm{Pb}}} \cdot 18.7488 \nonumber \\
    &= 0.1535\>\frac{\rm{MeV}\,\rm{cm}^2}{\rm{mol}}\cdot 11.34\>\frac{\rm{g}}{{\rm{cm}}^3}\cdot\frac{82}{207.2}\>\frac{\rm{mol}}{\rm{g}}\cdot 18.7488 \nonumber \\
    &= 12.9\>\frac{\rm{MeV}}{\rm{cm}}\>, \\
    -\left\langle \frac{\dd E}{\dd x} \right\rangle _{\rm{F}} &= \Xi\>\rho_{\rm{F}} \frac{Z_{\rm{F}}}{A_{\rm{F}}}\frac{1}{0.95443^2}\left[\ln\left(\frac{(10.5\cdot10^6\>\rm{eV})^2}{(115\>\rm{eV})^2}\right) - 2\cdot (0.95443)^2\right] \nonumber \\
    &= \Xi\>\rho_{\rm{F}} \frac{Z_{\rm{F}}}{A_{\rm{F}}} \cdot 23.0773 \nonumber \\
    &= 0.1535\>\frac{\rm{MeV}\,\rm{cm}^2}{\rm{mol}}\cdot 0.001696\>\frac{\rm{g}}{{\rm{cm}}^3}\cdot\frac{9}{19}\>\frac{\rm{mol}}{\rm{g}}\cdot 23.0773 \nonumber \\
    &= 0.002846\>\frac{\rm{MeV}}{\rm{cm}}\>.
\end{flalign}
%
Now all that is left is to compute the weighted sum as stated in (\ref{eq:bb-compound}):
%
\begin{flalign}
    -\left\langle \frac{\dd E}{\dd x} \right\rangle _{\text{compound}} &= -\frac{\rho\cdot w_{\rm{Pb}}}{\rho_{\rm{Pb}}}\cdot\left\langle \frac{\dd E}{\dd x} \right\rangle _{\rm{Pb}} - \frac{\rho\cdot w_{\rm{F}}}{\rho_{\rm{F}}}\cdot\left\langle \frac{\dd E}{\dd x} \right\rangle _{\rm{F}} \nonumber \\
    &= \frac{7.77\>\dfrac{\rm{g}}{\rm{cm}^3}\cdot 0.845}{11.34\>\dfrac{\rm{g}}{\rm{cm}^3}}\cdot 12.9\>\frac{\rm{MeV}}{\rm{cm}} + \frac{7.77\>\dfrac{\rm{g}}{\rm{cm}^3}\cdot 0.154}{0.001696\>\dfrac{\rm{g}}{\rm{cm}^3}}\cdot 0.002846\>\frac{\rm{MeV}}{\rm{cm}} \nonumber \\
    &= 9.47 \>\frac{\rm{MeV}}{\rm{cm}}\>.
    \label{eq:1sub3-result}
\end{flalign}
%
\subsection{Landau Distribution}
For detectors of moderate thickness, which we can consider as \textbf{thin absorbers}, we can use a highly-skewed 
Landau-Vavilov distribution to describe the energy loss of particles. The most probable energy loss $\Delta_p$ is given as:
%
\begin{equation}
    \bar{\Delta} = \Delta_p = \xi\left[\ln{\frac{2m_ec^2\beta^2\gamma^2}{I}} + \ln{\frac{\xi}{I} + j - \beta^2 - \delta(\beta\gamma)}\right]\>,
    \label{eq:landau}
\end{equation}
%
where $\delta(\beta\gamma)$ represents corrections due to the density effect, $j=0.200$ and $\xi$ is given as:
%
\begin{equation}
    \xi = \frac{K}{2}\left\langle\frac{Z}{A}\right\rangle\frac{x}{\beta^2}\>\rm{MeV}\>,
    \label{eq:xi}
\end{equation}
%
for $x$ in $\rm{g/cm}^2$ and $K = 0.3\,\rm{MeV}\,\rm{cm}^2\rm{/g}$. \textbf{Warning:} $x$ here is normalized with density $\rm{g/cm}^2$. This
means that $x = \rho\cdot d$ where $d$ is the thickness of the detector. To know which distribution to use, we can use the 
following rule of thumb:
%
\begin{equation}
    \kappa = \frac{\bar{\Delta}}{W_{\rm{max}}} \begin{cases}
        > 10 & \text{use Bethe-Bloch}\>, \\
        < 0.01 & \text{use Landau}\>.
    \end{cases}
\end{equation}
%
To determine the energy resolution of such a detector we can use the following formula:
%
\begin{equation}
    R_{\rm{FWHM}} = \frac{4\xi}{\Delta_p}\>.
    \label{eq:landau-res}
\end{equation}
%
\subsubsection{What is the Most Probable Energy Loss of a Charged Pion in Silicon?}
Let's calculate the most probable energy loss of a charged pion with a rest mass of $139.57\,\rm{MeV}$ and momentum of $0.5\,\rm{GeV}$ in a silicon based
detector which has a $320\,\mu\rm{m}$ thick silicon layer. Silicon has the following properties:
%
\begin{align*}
    & \rho = 2.32\>\rm{g/cm}^3\>, \\
    & Z = 14\>, \\
    & A = 28\>\rm{g/mol}\>.
\end{align*}
%
As before we first calculate the velocity $\beta$ and the Lorentz factor $\gamma$ for pions. Using (\ref{eq:beta}) we get:
%
\begin{equation}
    \beta = \frac{0.5\>\dfrac{\rm{GeV}}{c}\>c}{\sqrt{\left(0.5\>\dfrac{\rm{GeV}}{c}\>c\right)^2 + \left(0.13957\>\dfrac{\rm{GeV}}{c^2}\>c^2\right)^2}} \approx  0.96318\>,
\end{equation}
%
which gives us a Lorentz factor of:
%
\begin{equation}
    \gamma = \frac{1}{\sqrt{1-(0.96318)^2}} \approx 3.72\>.
\end{equation}
%
Likewise as before we want to calculate the mean excitation energy $I$ using (\ref{eq:I}) for $Z \ge 13$:
%
\begin{equation}
    I = 14\left(9.76 + 58.8\cdot 14^{-1.19}\right)\>\rm{eV} = 172.3\>\rm{eV}\>.
\end{equation}
%
We can also calculate our approximation for the maximum energy transfer in a single collision $W_{\text{max}}$ using (\ref{eq:wmax}), since we 
can spot it in the Landau distribution (\ref{eq:landau}):
%
\begin{equation}
    W_{\text{max}} = 2\cdot 0.511\>\frac{\rm{MeV}}{c^2}\>c^2\cdot(0.96318)^2(3.72)^2 = 13.12\>\rm{MeV}\>.
\end{equation}
Next we calculate $\xi$ using (\ref{eq:xi}):
%
\begin{equation}
    \xi = \frac{0.3\>\dfrac{\rm{MeV}\,\rm{cm}^2}{\rm{g}}}{2}\>\frac{14}{28}\>\frac{320\cdot10^{-4}\>\rm{cm}\cdot 2.32\>\dfrac{\rm{g}}{\rm{cm}^3}}{(0.96318)^2} = 0.0060\>\rm{MeV}\>.
\end{equation}
%
Now we can calculate the most probable energy loss using the Landau distribution (\ref{eq:landau}):
%
\begin{flalign}
    \Delta_p &= 0.006\>\rm{MeV}\,\left[\ln{\frac{13.12\cdot10^6\>\rm{eV}}{172.3\>\rm{eV}}} + \ln{\frac{0.006\cdot10^6\>\rm{eV}}{172.3\>\rm{eV}}} + 0.2 - (0.96318)^2\right] \nonumber \\
    &= 0.0844\>\rm{MeV}\>.
\end{flalign}
%
From this we can now also calculate the energy resolution of the detector using the formula (\ref{eq:landau-res}):
%
\begin{equation}
    R_{\rm{FWHM}} = \frac{4\cdot 0.006\>\rm{MeV}}{0.0844\>\rm{MeV}} = 28.5\%\>.
    \label{eq:2sub1-result}
\end{equation}
%
If we're paranoid if we've used the right distribution, we can calculate $\kappa$ as:
%
\begin{equation}
    \kappa = \frac{0.0844\>\rm{MeV}}{13.12\>\rm{MeV}} = 0.0064\>,
\end{equation}
%
which is less than $0.01$ so we've used the right distribution. Alternatively if we magically procure the 
result from the Bethe-Bloch equation, we'd get $\bar{\Delta} = 126\,\rm{keV}$ and $\sigma = 402\,\rm{keV}$ which would give
us $\kappa = 0.0105$ which still hints that we should use the Landau distribution.

\subsection{Cherenkov Radiation}
Charged particles moving through a medium with a velocity greater than the speed of light in that medium emit Cherenkov 
radiation. The angle of the emitted radiation is given by the Cherenkov angle which is defined as:
%
\begin{equation}
    \cos{\theta} = \frac{1}{\beta n}\>,
    \label{eq:cherenkov-angle}
\end{equation}
%
where $n$ is the refractive index of the medium. The threshold velocity for Cherenkov radiation is given as:
%
\begin{equation}
    \beta_{\text{thr}} = \frac{1}{n}\>.
    \label{eq:cherenkov-threshold}
\end{equation}
%
We can calculate the number of produced Cherenkov photons per unit length $x$ with the following formula:
%
\begin{equation}
    \frac{\dd^2 N}{\dd E\,\dd x} = \frac{\alpha z^2}{\hbar c}\sin^2{\theta}\>,
\end{equation}
%
which we can approximate for $z=1$ as:
%
\begin{equation}
    \frac{\dd^2 N}{\dd E\,\dd x} = \frac{370}{\rm{eV}\>\rm{cm}}\sin^2{\theta}\>.
    \label{eq:cherenkov-diff}
\end{equation}
%
if we assume that $\beta\approx\text{const.}$ and that the refractive index is not a function of the wavelength $n\neq n(\lambda)$.
Thus if we'd like to calculate the total number of Cherenkov photons produced in our detector we can use the following formula:
%
\begin{equation}
    N_{\rm{Cherenkov}} = \frac{370}{\rm{eV}\,\rm{cm}}\,\Delta x\,\Delta E\left(1 - \frac{1}{\beta^2 n^2}\right)\>,
    \label{eq:cherenkov-total}
\end{equation}
%
where $\Delta x$ is the thickness of the detector and $\Delta E$ is the energy of an emitted Cherenkov photon which is simply
calculated as:
%
\begin{equation}
    \Delta E = h\nu = \frac{hc}{\lambda} = \frac{1240\>\rm{eV}\,\rm{nm}}{\lambda}\>,
    \label{eq:cherenkov-energy}
\end{equation}
%
where $\lambda$ is the wavelength of the emitted Cherenkov photon.

\subsubsection{How many Cherenkov Photons are Produced in Water by a Proton?}
Let's calculate the number of Cherenkov photons produced in $1\,\rm{cm}$ of water by a proton with a rest mass of $0.938\,\rm{GeV}$ and a
momentum of $2\,\rm{GeV}$. The refractive index of water is $n = 1.33$. How many photons are detected with a photodetector which is sensitive to 
light between $250\,\rm{nm}$ and $800\,\rm{nm}$ with an average efficiency of $10\%$? \\

Using the formula (\ref{eq:beta}) we can calculate the velocity $\beta$ for the proton:
%
\begin{equation}
    \beta = \frac{2\>\dfrac{\rm{GeV}}{c}\>c}{\sqrt{\left(2\>\dfrac{\rm{GeV}}{c}\>c\right)^2 + \left(0.938\>\dfrac{\rm{GeV}}{c^2}\>c^2\right)^2}} \approx 0.90537\>.
\end{equation}
%
The Cherenkov angle $\theta$ can be calculated using the formula (\ref{eq:cherenkov-angle}):
%
\begin{equation}
    \theta = \arccos{\frac{1}{0.90537\cdot 1.33}} = 33.85^\circ\>.
\end{equation}
%
From this we can calculate the number of Cherenkov photons produced per unit length using the formula (\ref{eq:cherenkov-diff}):
%
\begin{equation}
    \frac{\dd^2 N}{\dd E\,\dd x} = \frac{370}{\rm{eV}\>\rm{cm}}\sin^2{(33.85^\circ)} = 114.8\>\rm{eV}^{-1}\rm{cm}^{-1}\>.
\end{equation}
%
Now we can calculate the total number of Cherenkov photons produced in $1\,\rm{cm}$ of water using the formula (\ref{eq:cherenkov-total}):
%
\begin{align}
    N_{\rm{min}} &= \frac{114.8}{\rm{eV}\>\rm{cm}}\cdot 1\>\rm{cm}\cdot \frac{1240\>\rm{eV}\,\rm{nm}}{250\>\rm{nm}} = 569.4\>, \\
    N_{\rm{max}} &= \frac{114.8}{\rm{eV}\>\rm{cm}}\cdot 1\>\rm{cm}\cdot \frac{1240\>\rm{eV}\,\rm{nm}}{800\>\rm{nm}} = 177.9\>.
\end{align}
%
Using these two values we can very roughly estimate the number of detected Cherenkov photons in the range $250\,\rm{nm}$ to $800\,\rm{nm}$.
Essentially we use a linear approximation of the integral of the number of Cherenkov photons produced per unit length over the range of wavelengths and 
multiply it by the efficiency of the photodetector. Thus the number of detected Cherenkov photons is:
%
\begin{equation}
    N_{\rm{det}} = 0.1\cdot (N_{\rm{max}} - N_{\rm{min}}) \approx 39\>.
\end{equation}

\subsection{Neutrinos and Dark Matter}
Neutrinos are elementary particles that interact only very weakly with matter. They are produced in nuclear reactions and in the decay of
particles. Neutrinos are classified into three types: electron neutrinos $\nu_e$, muon neutrinos $\nu_\mu$ and tau neutrinos $\nu_\tau$.
There are three possible interactions we can detect:
%
\begin{align*}
    & \nu_x + e^- \rightarrow \nu_x + e^-\> \qquad \text{elastic scattering}\>, \\
    & \nu_e + n \rightarrow p + e^-\> \qquad \text{charged current interaction}\>, \\
    & \bar{\nu}_e + p \rightarrow n + e^+\> \qquad \text{inverse beta decay}\>.
\end{align*}
%
All of these interactions produce electrons/positrons which emit Cherenkov radiation. For an incoming flux of neutrinos $F$ we can calculate the interaction 
cross-section as:
%
\begin{equation}
    \frac{\dd\sigma}{\dd\Omega} = F\>\frac{\dd N}{\dd\Omega}\>,\
    \label{eq:cross-section}
\end{equation}
%
where $\frac{\dd N}{\dd\Omega}$ is the number of Cherenkov photons produced per unit angle. We often measure cross-sections in barns where $1\,\rm{b} = 10^{-28}\,\rm{m}^2$.
For neutrinos that only interact via the weak force the interaction cross-section is about:
%
\begin{equation}
    \bm{\sigma_\nu} = \bm{10^{-20}}\,\textbf{b} = \bm{10^{-44}}\,\textbf{cm}\bm{^2} = \bm{10^{-48}}\,\textbf{m}\bm{^2}\>.
\end{equation}
%
The number of interactions can be calculated as:
%
\begin{equation}
    N = t\phi \sigma_\nu N_\text{sc}\>,
    \label{eq:nu-interactions}
\end{equation}
%
where $t$ is the time of exposure, $\phi$ is the flux of neutrinos, $\sigma_\nu$ is the interaction cross-section and $N_\text{sc}$ is the number of target particles (scattering 
centers). Both $\sigma_\nu$ and $N_\text{sc}$ are in general dependant on the type of interaction. We can calculate the number of number of scattering centers as:
%
\begin{equation}
    N_\text{sc} = \frac{m N_a}{A}\>,
\end{equation}
%
if we assume that the number of scattering centers is the same as the number of nucleons in the target material. 

\subsubsection{What mass of pure atomic hydrogen is needed to detect 1000 solar neutrinos per year?}
Lets assume a neutrino flux of $\phi = 6\cdot 10^{14}\,\rm{m}^{-2}\rm{s}^{-1}$, an interaction cross-section of $\sigma_\nu = 10^{-48}\,\rm{m}^2$ and that the 
number of scattering centers is the same as the number of nucleons in the hydrogen molecule. Thus the number of detected neutrinos per year is given as:
%
\begin{equation}
    N = t\phi\sigma\>\frac{mN_a}{A}\>.
\end{equation}
%
If we want to detect $1000$ neutrinos per year we can calculate the mass of hydrogen as:
%
\begin{flalign}
    m &= \frac{A}{N_a t\phi\sigma} \cdot 1000 \nonumber \\
    &= \frac{1\>\dfrac{\rm{g}}{\rm{mol}}}{\left[6.02\cdot10^{23}\>\dfrac{1}{\rm{mol}}\right]\left(3.15\cdot 10^7\>\rm{s}\right)\left[6\cdot 10^{14}\>\dfrac{1}{\rm{m}^2\rm{s}}\right]\left(10^{-48}\>\rm{m}^2\right)} \cdot 1000 \nonumber \\
    &= 87.9\>\rm{kg}\>.
\end{flalign}
%
\subsubsection{Detection of WIMP's in a germanium detector}
WIMP's (Weakly Interacting Massive Particles) are hypothetical particles that are thought to make up dark matter. Let's say that we 
are trying to detect them through elastic scattering with germanium nuclei. We take the mass of a WIMP to be $100\,\rm{GeV}$ and presume that 
they are stationary in intergalactic space. Our solar system is moving through intergalactic space at a velocity of $2.2\cdot 10^5\,\rm{m/s}$.
We first need to estimate the maximum energy transfer in one collision with a germanium nucleus from which we can then calculate the needed mass 
of germanium in the detector to get one event per year. The estimated cross-section for WIMP-nucleus scattering is
$\sigma_{\text{WIMP}} = 10^{-45}\,\rm{cm}^2$. The estimated flux of WIMP's is $\phi = 10^5\,\rm{cm}^{-2}\rm{s}^{-1}$. \\

We can directly use equation (\ref{eq:nu-interactions}) to calculate the mass of germanium needed in the detector:
%
\begin{flalign}
    m &= \frac{NA}{N_a t\phi\sigma} \nonumber \\
    &= \frac{1\cdot 72.6\>\dfrac{\rm{g}}{\rm{mol}}}{\left[6.02\cdot10^{23}\>\dfrac{1}{\rm{mol}}\right](3.15\cdot 10^7\>\rm{s})\left[10^5\>\dfrac{1}{\rm{cm}^2\rm{s}}\right]\left(10^{-45}\>\rm{cm}^2\right)} \nonumber \\
    &= 3.826 \cdot 10^7\>\rm{t} = 38\>\rm{kt}\>.
\end{flalign}
%
where we used the atomic mass of germanium $A = 72.6\,\rm{g/mol}$ and $N = 1$. To calculate the maximum energy transfer 
in one collision we'd need to calculate the kinematics of the problem along with taking into account conservation laws. This 
would be best done in the center of mass frame and then transformed back to the lab frame. This is a bit too much for this
exercise so we can try to estimate the maximum energy transfer semi-classically. The kinetic energy the germanium nucleon receives 
is:
%
\begin{equation}
    W_2 = \frac{4mM}{(m+M)^2}\>W_1\>,
\end{equation}
%
where $M$ is the mass of the WIMP, $m$ is the mass of the germanium nucleus and $W_1$ is the kinetic energy of the WIMP before the collision.
The germanium nucleus is made of $72$ nucleons and the mass of a nucleon is $\sim 1\,\rm{GeV}$. Thus the mass of the germanium nucleus is
$m = 72\,\rm{GeV}$. We can calculate $\beta$ for the WIMP as:
%
\begin{equation}
    \beta = \frac{v}{c} = \frac{2.2\cdot10^5\dfrac{\rm{m}}{\rm{s}}}{3\cdot10^8\dfrac{\rm{m}}{\rm{s}}} = 0.733\cdot10^{-3}\>.
\end{equation}
%
From this we can calculate the initial kinetic energy of the WIMP as:
%
\begin{equation}
    W_1 = \frac{1}{2}Mv^2 =  \frac{1}{2}Mc^2\beta^2 = 0.5\cdot 100\cdot 10^9\>\frac{\rm{eV}}{c^2}\>c^2\cdot(0.733\cdot10^{-3})^2 = 27.0\>\rm{keV}\>. 
\end{equation}
%
Thus the maximum energy transfer in one collision is:
%
\begin{flalign}
    W_2 &= \frac{4\cdot 72\cdot 100}{(72+100)^2}\>27.0\>\rm{keV} \nonumber \\
    &= 0.9735 \cdot 27.0\>\rm{keV} \nonumber \\
    &= 26.3\>\rm{keV}\>.
\end{flalign}
%

\section{Detectors}
\subsection{Semiconductor Detectors}
Semiconductor detectors are most commonly placed close to the interaction point of a collider or experiment since their 
purpose is to measure the positions of various output particles without disturbing them. The most common semiconductor
detectors are silicon and germanium detectors. They are made of a p-n junction which is reverse biased. When a charged particle
passes through the detector it creates electron-hole pairs which are then separated by the electric field of the
reverse biased p-n junction. The electrons and holes are then collected at the electrodes of the detector. The average number 
of detected electron-hole pairs is given as:
%
\begin{equation}
    \left\langle N \right\rangle = \frac{\Delta E}{w}\>,
    \label{eq:semi-pairs}
\end{equation}
%
where $\Delta E$ is the energy deposited in the detector and $w$ is the energy needed to create an electron-hole pair in the 
semiconductor. Some common values for $w$ are:
%
\begin{align*}
    & w_{\text{Si}}(300\,\rm{K}) = 3.6\,\rm{eV}\>, \\
    & w_{\text{Si}}(77\,\rm{K}) = 3.7\,\rm{eV}\>, \\
    & w_{\text{Ge}}(77\,\rm{K}) = 2.9\,\rm{eV}\>, \\
\end{align*}
%
The energy resolution of a semiconductor detector is given as:
%
\begin{equation}
    R = \frac{\sigma_N}{\left\langle N \right\rangle} = \frac{\sqrt{F\cdot\left\langle N \right\rangle}}{\left\langle N \right\rangle}\>,
    \label{eq:semi-res-full}
\end{equation}
%
where $F$ is the Fano factor which is a measure of the fluctuations in the number of detected electron-hole pairs. The Fano factor
is usually around:
%
\begin{equation}
    F = \begin{cases}
        0.086\>-\>0.16 & \text{for silicon}\>, \\
        0.06\>-\>0.13 & \text{for germanium}\>.
    \end{cases}
\end{equation}
%
Using the equation (\ref{eq:semi-pairs}) we can calculate the energy resolution directly as:
%
\begin{equation}
    R = \sqrt{\frac{F\cdot w}{\Delta E}} = \frac{R_\text{FWHM}}{2.35}\>,
    \label{eq:semi-res}
\end{equation}
%
where $R_\text{FWHM}$ is the full width at half maximum of the energy resolution.

\subsubsection{What is the energy resolution of a silicon detector at $300\,\rm{K}$?}
Consider a detector with a $1\,\rm{mm}$ thick silicon layer at $300\,\rm{K}$. In the case of a perpendicularly crossing 
kaon with a rest mass of $0.493\,\rm{GeV}$ and momentum of $4\,\rm{GeV}$ what is the energy resolution? Silicon has the following 
properties:
%
\begin{align*}
    & w_{\text{Si}}(300\,\rm{K}) = 3.6\,\rm{eV}\>, \\
    & \rho = 2.32\>\rm{g/cm}^3\>, \\
    & Z = 14\>, \\
    & A = 28\>\rm{g/mol}\>.
\end{align*}
%
We can calculate the energy loss of the kaon in the silicon detector using the Landau distribution (\ref{eq:landau}). Before that 
we need to calculate the velocity $\beta$ and the Lorentz factor $\gamma$ for the kaon. Using (\ref{eq:beta}) we get:
%
\begin{equation}
    \beta = \frac{4\>\dfrac{\rm{GeV}}{c}\>c}{\sqrt{\left(4\>\dfrac{\rm{GeV}}{c}\>c\right)^2 + \left(0.493\>\dfrac{\rm{GeV}}{c^2}\>c^2\right)^2}} \approx 0.99249\>,
\end{equation}
%
which gives us a Lorentz factor of:
%
\begin{equation}
    \gamma = \frac{1}{\sqrt{1-(0.99249)^2}} \approx 8.175\>.
\end{equation}
%
Next we need to calculate the mean excitation energy $I$ using (\ref{eq:I}) for $Z \ge 13$:
%
\begin{equation}
    I = 14\left(9.76 + 58.8 \cdot 14^{-1.19}\right)\>\rm{eV} = 172.1\>\rm{eV}\>,
\end{equation}
%
and the maximum energy transfer in a single collision $W_{\text{max}}$ using (\ref{eq:wmax}):
%
\begin{equation}
    W_{\text{max}} = 2\cdot 0.511\>\frac{\rm{MeV}}{c^2}\>c^2\cdot(0.99249)^2(8.175)^2 = 67.3\>\rm{MeV}\>.
\end{equation}
%
We also need to calculate $\xi$ using (\ref{eq:xi}):
%
\begin{flalign}
    \xi &= \frac{0.3\>\dfrac{\rm{MeV}\,\rm{cm}^2}{\rm{g}}}{2}\>\frac{14}{28}\>\frac{0.1\>\rm{cm}\cdot 2.32\>\dfrac{\rm{g}}{\rm{cm}^3}}{(0.99249)^2} \nonumber \\
    &= 0.0178\>\rm{MeV}\>.
\end{flalign}
%
Now all that is left to get the most probable energy loss is to use the Landau distribution (\ref{eq:landau}):
%
\begin{flalign}
    \Delta_p &= 0.0178\>\rm{MeV}\,\left[\ln{\frac{67.3\cdot10^6\>\rm{eV}}{172.1\>\rm{eV}}} + \ln{\frac{0.0178\cdot10^6\>\rm{eV}}{172.1\>\rm{eV}}} + 0.2 - (0.99249)^2\right] \nonumber \\
    &= 0.298\>\rm{MeV}\>.
\end{flalign}
%
From this we can calculate the energy resolution of the detector using the formula (\ref{eq:semi-res}):
%
\begin{flalign}
    R &= \sqrt{\frac{0.086\cdot 3.6\>\rm{eV}}{0.298\cdot 10^6\>\rm{eV}}}\nonumber \\
    &= 0.11\%\>, \\
    R_\text{FWHM} &= 2.35\cdot 0.11\% = 0.26\%\>.
\end{flalign}
%
For the purpose of education if we were to repeat the entire calculation while taking values for germanium we'd get:
%
\begin{align*}
    &I = 342\>\rm{eV}\>, \\
    &\xi = 35.77\>\rm{keV}\>, \\
    &\Delta_p = 573\>\rm{keV}\>, 
\end{align*}
%
which would give us an energy resolution of $R_\text{Ge} = 0.072\%$ and $R_{\text{FWHM}_\text{Ge}} = 0.17\%$.

\subsubsection{What voltage is needed to get a $1\,\rm{mm}$ thick depletion region in the detector from the previous exercise?}
Let's say that the impurity concentration is $N_D = 6\cdot 10^{14}\,\rm{cm}^{-3}$ and the dielectric constant of silicon is $\varepsilon = 11.7$.
The equation for the depletion region width is:
%
\begin{equation}
    d = \sqrt{\frac{2\varepsilon\varepsilon_0 U}{e_0 N_D}}\>.
    \label{eq:depletion-thickness}
\end{equation}
%
We can use this equation to find the voltage $U$ needed to get a $1\,\rm{mm}$ thick depletion region:
%
\begin{flalign}
    U &= \frac{d e_0 N_D}{2\varepsilon\varepsilon_0} \nonumber \\
    &= \frac{\left(0.1\>\rm{cm}\right)^2\left[1.6\cdot10^{-19}\>\rm{As}\right]\left(6\cdot10^{14}\dfrac{1}{\rm{cm}^3}\right)}{2\cdot11.7\left(8.85\cdot10^{-12}\cdot 10^2 \>\dfrac{\rm{As}}{\rm{V}\,\rm{cm}}\right)} \nonumber \\
    &= 45.4\>\rm{V}\>.
\end{flalign}

\subsection{Ionization Detectors}
\paragraph{Ionization Chambers}
Gas ionization chambers are detectors that are filled with a gas and have an electric field applied to them. When a charged particle
passes through the gas it ionizes the gas atoms and the electrons and ions are collected at the electrodes of the detector. Without multiplication of the signal 
the energy resolution is given as:
%
\begin{equation}
    R = \sqrt{\frac{F\cdot w}{\Delta E}}\>,
    \label{eq:ionization-res}
\end{equation}
%
where $F$ is the Fano factor which is around $F = 0.2$ for gas ionization chambers and where we already took into account 
the number of created electron-ion pairs and that they are Poisson distributed:
%
\begin{align}
    N &= \frac{\Delta E}{w}\>, \\
    \label{eq:ion-N}
    \sigma_N = \sqrt{F\cdot N}\>. \\
    \label{eq:ion-sigma} 
\end{align}

The Fano factor is needed to correct the fact that subsequent electron-ion pairs are not entirely statistically 
independent. $w$ is the energy needed to create an electron-ion pair in the gas.

\paragraph{Proportional Counters}
Proportional counters are gas ionization chambers with a multiplication factor.
Since the created electron-ion pairs are hard to detect we use the Townsend avalanche effect to amplify 
the signal, which gives us a cascade of electron-ion pairs. 
For argon $Ar$ it is $w_\text{Ar} = 26\,\rm{eV}$. The amount of charge we collect is:
%
\begin{align}
    Q &= N\cdot e \qquad \text{without multiplication}\>, \\
    Q_\text{mult} &= N\cdot e\cdot M \qquad \text{with multiplication}\>,
    \label{eq:ionization-charge}
\end{align}
%
where $M$ is the multiplication factor. The energy resolution of a gas ionization chamber with multiplication 
is \textbf{lower} than without multiplication. The energy resolution with multiplication is given as:
%
\begin{equation}
    R_\text{mult} = \sqrt{\frac{w\left(F+b\right)}{\Delta E}}\>,
    \label{eq:ionization-res-mult}
\end{equation}
%
where $b$ is a constant that depends on the detector, usually between $0.4$ and $0.7$.

\subsubsection{What is the energy resolution of a gas ionization chamber?}
What is the energy resolution of an ionization chamber with a thickness of $d=10\,\rm{cm}$, for a MIP particle if the gas 
used is the so-called \emph{magic gas} which has the properties:
%
\begin{align*}
    & Q_\text{Ar} = 75\%\>, \\
    & Q_{\rm{C}_4 \rm{H}_{10}} = 25\%\>, \\
    &\rho_\text{Ar} = 1.66\>\rm{g/L}\>, \\
    &\rho_{\rm{C}_4 \rm{H}_{10}} = 2.5\>\rm{g/L}\>, \\
    & w_\text{Ar} = 26\,\rm{eV}\>, \\
    & w_{\rm{C}_4 \rm{H}_{10}} = 23\,\rm{eV}\>, \\
    & F_\text{Ar} = 0.2\>, \\
    & F_{\rm{C}_4 \rm{H}_{10}} \approx 0.2\>, 
\end{align*}
%
where $C_4H_{10}$ is isobutane and $Q$ represents percentage by volume in the gas mixture. Since we're dealing with a MIP (Minimum Ionizing Particle) we can assume that the energy 
deposited is:
%
\begin{equation}
    - \frac{\dd E}{\dd x} = 2 \frac{\rm{MeV}\>\rm{cm}^2}{\rm{g}}
\end{equation}
%
We need to calculate the deposited energy by the individual gas components. So the energy deposited in the argon is:
%
\begin{flalign}
    \Delta E_\text{Ar} &= Q_\text{Ar}\cdot \rho_\text{Ar}\cdot d \left(- \frac{\dd E}{\dd x}\right) \nonumber \\
    &= 0.75\cdot 1.66\cdot10^{-3}\>\frac{\rm{g}}{\rm{cm}^3}\cdot 10\>\rm{cm}\cdot 2\>\frac{\rm{MeV}\>\rm{cm}^2}{\rm{g}} \nonumber \\
    &= 0.0249\>\rm{MeV}\>.
\end{flalign}
%
Likewise for isobutane we get:
%
\begin{flalign}
    \Delta E_{\rm{C}_4 \rm{H}_{10}} &= Q_{\rm{C}_4 \rm{H}_{10}}\cdot \rho_{\rm{C}_4 \rm{H}_{10}}\cdot d \left(- \frac{\dd E}{\dd x}\right) \nonumber \\
    &= 0.25\cdot 2.5\cdot10^{-3}\>\frac{\rm{g}}{\rm{cm}^3}\cdot 10\>\rm{cm}\cdot 2\>\frac{\rm{MeV}\>\rm{cm}^2}{\rm{g}} \nonumber \\
    &= 0.014\>\rm{MeV}\>.
\end{flalign}
%
Now we can calculate the number of created electron-ion pairs for argon and isobutane using the formula (\ref{eq:ion-N}):
%
\begin{align}
    &N_\text{Ar} = \frac{0.0249\cdot10^6\>\rm{eV}}{26\>\rm{eV}} = 958\>, \\
    &N_{\rm{C}_4 \rm{H}_{10}} = \frac{0.014\cdot10^6\>\rm{eV}}{23\>\rm{eV}} = 609\>,
\end{align}
%
from which we can calculate the standard deviation using the formula (\ref{eq:ion-sigma}):
%
\begin{align}
    &\sigma_{N_\text{Ar}} = \sqrt{0.2\cdot 958} = 13.84\>, \\
    &\sigma_{N_{\rm{C}_4 \rm{H}_{10}}} = \sqrt{0.2\cdot 609} = 11.03\>.
\end{align}
%
The trick here is how to combine the two deviations and number of created electron-ion pairs. We know from statistics that 
we can sum the squares of the deviations and then take the square root of the sum to get the total deviation. The number of pairs 
is simply the sum of the number of pairs created in argon and isobutane. From this we can calculate the energy resolution as:
%
\begin{flalign}
    R &= \frac{\sqrt{{\sigma_{N_\text{Ar}}}^2+{\sigma_{N_{\rm{C}_4 \rm{H}_{10}}}}^2}}{N_\text{Ar} + N_{\rm{C}_4 \rm{H}_{10}}} \nonumber \\
    &= \frac{\sqrt{\left(13.84\right)^2+\left(11.03\right)^2}}{958 + 609} \nonumber \\
    &= 1.1\%\>.
\end{flalign}
%
\subsubsection{What is the energy resolution of a proportional counter?}
For educational purposes let's calculate how the energy resolution worsens with multiplication in a 
proportional counter. Let's keep the rest of the data as in the previous exercise and assume that the multiplication
factor is $M = 900$ and that the constant $b = 0.5$. To get the new energy resolution we can recycle our previous result and add 
an additional term to the resolution that is due to multiplication. The combined resolution is:
%
\begin{equation}
    R = \sqrt{R_N^2 + R_M^2}\>,
\end{equation}
%
where $R_N$ is the resolution without multiplication and $R_M$ is the resolution decrease due to multiplication. The resolution
decrease due to multiplication is given as:
%
\begin{flalign}
    R_M &= \sqrt{\frac{b}{N}} \nonumber \\
    &= \sqrt{\frac{0.5}{958 + 609}} \nonumber \\
    &= 1.8\%\>,
\end{flalign}
%
where we took the \textbf{primary number of created pairs}, not the number of pairs after multiplication. Thus the total resolution is:
%
\begin{equation}
    R = \sqrt{(0.011)^2 + (0.018)^2} = 2.9\%\>.
\end{equation}

\subsubsection{Why the cylindrical geometry? What voltage would be needed to achieve the same electric field in a parallel plate capacitor?}
Let's imagine that the structure of our detector is analogous to a Geiger-Muller tube, so a cylinder with a wire in the middle. This is only an approximation
since actual ionization chambers are more complex and are often made of thousands of parallel wires inside a cylindrical gas chamber. The reason for the cylindrical
geometry is exactly the properties of the electric field. Taking the center wire thickness to be $a=0.008\,\rm{cm}$ and the radius of the cylinder to be 
$b=1\,\rm{cm}$, the electric field inside the cylinder is given as:
%
\begin{equation}
    E(r) = \frac{U}{r\ln{\frac{b}{a}}}\>,
\end{equation}
%
which at $U=2000\,\rm{V}$ yields an electric field at the center wire of the cylinder:
%
\begin{flalign}
    E(r=b) &= \frac{2000\>\rm{V}}{1\>\rm{cm}\>\ln{\frac{1\>\rm{cm}}{0.008\>\rm{cm}}}} \nonumber \\
    &= 5.2\cdot 10^6\>\rm{V/m}\>.
\end{flalign}
%
In comparison, if we wanted to achieve the same electric field inside a planar capacitor with a distance of $d=1\,\rm{cm}$ we'd need a voltage of:
%
\begin{equation}
    U = E\cdot d = 5.2\cdot 10^6\>\rm{V/m}\cdot 1\>\rm{cm} = 52\>\rm{kV}\>.
\end{equation}
%
A power supply that can provide $52\,\rm{kV}$ is much more expensive and harder to maintain than a power supply that can provide $2\,\rm{kV}$, hence the 
practical choice of a cylindrical geometry.

\subsection{Scintillation Detectors}
In scintillation detectors the energy of a particle is converted into light. This happens when the particle interacts with the scintillator and
excites the atoms in the scintillator. The excited atoms then de-excite through an intermediate state and emit light. The light is then collected by a photomultiplier tube (PMT)
which converts the light into an electrical signal. The intermediate state is important since it allows for the scintillator to emit light out of the crystal. If it were not so the 
emitted photons would have enough energy to one again excite the atoms in the scintillator. We have two types of scintillators: organic and inorganic. Organic scintillators are
made of organic compounds and are usually liquid or plastic. Inorganic scintillators are made of inorganic compounds and are usually crystals. We can the energy needed for the creation of 
one photon for different scintillators:
%
\begin{align*}
    &\text{Organic}\quad \begin{cases}
        & w_{\text{Anthracene}} = 60\>\rm{eV}\>, \\
        & w_{\text{Plastic}} = 25\>\rm{eV}\>, \\
    \end{cases}
    \quad \\
    &\text{Inorganic}\quad \begin{cases}
        & w_{\text{NaI(Tl)}} = 100\>\rm{eV}\>, \\
        & w_{\text{BGO}} = 300\>\rm{eV}\>. \\
    \end{cases}
\end{align*}
%
A more commonly used property of scintillators is their \textbf{light yield} which is the number of photons produced per unit energy, like so:
%
\begin{equation}
    \rm{Ly} = \frac{N\>\text{photons}}{\Delta E\>\text{energy deposited by particle}}\>.
    \label{eq:light-yield}
\end{equation}
%
The light yield is usually given in $\rm{photons/MeV}$. The energy resolution of a scintillation detector is given as:
%
\begin{equation}
    R = \frac{\sigma_N}{N} = \frac{\sqrt{N}}{N} = \frac{1}{\sqrt{N}}\>,
    \label{eq:scint-res}
\end{equation}
%
where $N$ is the number of scintillation photons, which is calculated simply by:
%
\begin{equation}
    N = \rm{Ly}\cdot\Delta E = \frac{\Delta E}{w}\>.
    \label{scint-N}
\end{equation}
%
Sometimes we also se the use of the efficiency of the scintillator which is defined as:
%
\begin{equation}
    \varepsilon = \frac{E_\gamma}{\Delta E} = \frac{N\cdot h\nu}{\Delta E}\>.
    \label{eq:scint-eff}
\end{equation}
%
\subsubsection{How many scintillation photons are produed in a $\rm{CsI(Tl)}$ scintillator?}
Our electromagnetic calorimeter uses $30\,\rm{cm}$ long $\rm{CsI(Tl)}$ crystals. If a particle deposits $4\,\rm{GeV}$ of 
energy in the crystal how many scintillation photons are produced? The light yield of $\rm{CsI(Tl)}$ is $60000\,\rm{photons/MeV}$.
We get the number of produced photons using the formula (\ref{scint-N}):
%
\begin{equation}
    N = 60000\>\frac{\rm{photons}}{\rm{MeV}}\cdot 4\cdot10^3\>\rm{MeV} = 240\cdot10^6\>\rm{photons}\>.
\end{equation}
%
This theoretically gives us an energy resolution of:
%
\begin{equation}
    R = \frac{1}{\sqrt{240\cdot10^6}} = 0.0065\%\>,
\end{equation}
%
however the actual value is higher due to the quantum efficiency $\bar{\rm{QE}}$ of the photomultiplier tube and the actual efficiency of 
collection of the scintillation photons $\varepsilon_{\text{coll}}$. We take an average value for the quantum efficiency
as it is generally dependent on the wavelength of the light. If we take $\bar{\rm{QE}} = 0.2$ and $\varepsilon_{\text{coll}} = 0.7$ we get the 
new corrected number of detected particles as:
%
\begin{equation}
    N_{\text{det}} =N\cdot\bar{\rm{QE}}\cdot\varepsilon_\text{coll}=240\cdot10^6\cdot 0.2\cdot 0.7 = 33.6\cdot10^6\>.
\end{equation}
%
\begin{equation}
    R_{\text{eff}} = \frac{1}{\sqrt{240\cdot10^6\cdot 0.2\cdot 0.7}} = 0.017\%\>.
\end{equation}
%
\paragraph{Use of Scintillators in Nuclear Medicine}
\subsubsection{What is the Resolution of a SPECT Camera?}
A SPECT camera is a Single Photon Emission Computed Tomography camera. It is used in nuclear medicine to detect gamma rays. 
The camera uses a $\rm{NaI(Tl)}$ scintillator with a light yield of $40000\,\rm{photons/MeV}$. The incoming gamma rays have an energy of 
$140.5\,\rm{keV}$. We can calculate the number of produced scintillation photons using the formula (\ref{scint-N}):
%
\begin{equation}
    N = 40000\>\frac{\rm{photons}}{\rm{MeV}}\cdot 140.5\cdot10^{-3}\>\rm{MeV} = 5620\>\rm{photons}\>.
\end{equation}
%
The energy resolution of the camera is then:
%
\begin{equation}
    R = \frac{1}{\sqrt{5620}} = 1.33\%\>.
\end{equation}

\subsubsection{What is the Resolution of a PET Camera with a $\rm{BGO}$ scintillator?}
A PET camera is a Positron Emission Tomography camera. It is used in nuclear medicine to detect positron annihilation gamma rays. An interesting 
practical use of anti-matter by the way. Let's say we don't have the light yield of $\rm{BGO}$ but we know that the energy needed to create one photon
is $w=300\,\rm{eV}$. The incoming gamma rays have an energy of $511\,\rm{keV}$. We can calculate the number of produced scintillation photons 
using the formula (\ref{scint-N}):
%
\begin{equation}
    N = \frac{511\cdot10^{3}\>\rm{eV}}{300\>\rm{eV}} = 1703\>\rm{photons}\>,
\end{equation}
%
from which we can simply use the formula (\ref{eq:scint-res}) to get the energy resolution:
%
\begin{equation}
    R = \frac{1}{\sqrt{1703}} = 2.4\%\>.
\end{equation}

\subsection{Particle Identification Detectors}
We can separate particles based on their interactions and their mass. We almost always have a strong magnetic field present
inside the detector/ This is so we can determine the momentum of the particle by measuring the curvature of the particle's trajectory.
The momentum in the transverse plane is given as:
%
\begin{equation}
    p_T = q\cdot B\cdot R\>,
    \label{eq:B-momentum}
\end{equation}
%
where $q$ is the charge of the particle, $B$ is the magnetic field and $R$ is the radius of curvature. 

% \newpage
% \bibliographystyle{unsrt}
% \bibliography{main}

% End document
\end{document}

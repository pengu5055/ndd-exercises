% This is a LaTeX template kindly taken from Jernej Debevec.
% Provided by Miha Muskinja for the purpose of the seminar I in the 1st year
% of the 2nd cycle of the study of physics at the Faculty of Mathematics and Physics, University of Ljubljana.

% Set the document class and options
\documentclass[10pt, titlepage, a4paper]{article}
\usepackage[a4paper, inner=2.5cm, outer=2.5cm, top=2.25cm, bottom=2.25cm]{geometry}
\usepackage{graphicx}
\usepackage{hyperref}
\usepackage{wrapfig}
\usepackage{amsmath}
\usepackage{amssymb}
\usepackage{bm}
\usepackage{float}
\usepackage{xfrac}
\usepackage{accents}
\hypersetup{colorlinks=true}

% Load the natbib package for citation style
\usepackage{natbib}

% Some macros for commonly used symbols in physics/quantum mechanics
\newcommand{\bb}[1]{\bm#1}
\newcommand{\dd}{\mathrm{d}}
\newcommand{\pp}{\partial}
\newcommand{\dg}{\dagger}
\newcommand{\la}{\langle}
\newcommand{\ra}{\rangle}
\newcommand{\id}{\mathbb{1}}
\newcommand{\T}{\mathsf{T}}
\newcommand{\ua}{\uparrow\>}
\newcommand{\da}{\downarrow\>}
\newcommand{\fs}[1]{\slashed{#1}}  % Feynmann slash
\newcommand{\mc}[1]{\mathcal{#1}}
\newcommand\thickbar[1]{\accentset{\rule{.5em}{.03em}}{#1}}
\renewcommand{\bar}{\thickbar}

\numberwithin{equation}{section}

% Start the document
\begin{document}

% The title page
\begin{titlepage}
{\centering
\includegraphics[width=6cm]{logo_fmf.pdf}

\vspace{0.8cm}
{\small Department of Physics}

\vspace{5cm}
\vspace{0.5cm}
{\huge\textbf{Advanced Particle Detectors and Data Analysis}} \\
\vspace{0.5cm}
{\large\textbf{Notes for Exercises}}

\vfill
\textbf{Author:} Marko Urbanč \\
\textbf{Professor:} prof. dr. Peter Križan \\ 
\textbf{Assistant:} doc. dr. Rok Dolenec \\

\vspace{1cm}
Ljubljana, December 2024 \\
}
\vspace{3cm}
\end{titlepage}

% Add table of conents
\hypersetup{pageanchor=true}
\pagenumbering{roman}
\setcounter{page}{2}
\tableofcontents
\vspace{1cm}

% Proceed with the main body
\pagenumbering{arabic}

\section{Interactions of Particles with Photons}
\subsection{Bethe-Bloch Equation}
The Bethe-Bloch equation describes the mean energy loss per distance traveled while traversing through matter.
We generally use the Bethe-Bloch equation when we are dealing with \textbf{thick absorbers}, such as the ones in calorimeters.
Do note that the Bethe-Bloch equation does not accurately describe the energy loss of \textbf{electrons} and \textbf{positrons} 
due to their small mass and the fact that they suffer from much larger energy losses due to bremsstrahlung and pair production.
For a particle with charge $z$ and velocity $\beta = v/c$, the Bethe-Bloch equation is given as:
%
\begin{equation}
    -\left\langle \frac{\dd E}{\dd x} \right\rangle = 2\pi N_a r_e^2 m_e c^2 \rho \frac{Z}{A}\frac{z^2}{\beta^2}\left[\ln\left(\frac{2m_ec^2\beta^2\gamma^2W_{\text{max}}}{I^2}\right) - 2\beta^2 - \delta - 2\frac{C}{Z}\right]\>,
    \label{eq:bethe-bloch}
\end{equation}
%
where $\delta$ is the \textbf{density effect correction} and $C$ is the \textbf{shell correction}. The rest is as follows:
%
\begin{gather*}
    N_a = 6.022 \times 10^{23}\,\text{mol}^{-1}\>, \quad r_e = 2.818 \times 10^{-15}\,\text{m}\>, \quad m_e = 9.11 \times 10^{-31}\,\text{kg}\>, \quad c = 3 \times 10^8\,\text{m/s}\>, \nonumber \\
    \rho = \text{density of the material}\>, \quad A = \text{atomic mass of the material}\>, \quad Z = \text{atomic number of the material}\>, \\
    \quad \gamma = \frac{1}{\sqrt{1-\beta^2}}\>, \quad W_{\text{max}} = \text{maximum energy transfer in a single collision}\>, \quad I = \text{mean excitation energy}\>. \nonumber
\end{gather*}
%
The constant factor in the equation can be written as:
%
\begin{equation}
    \Xi = \bm{2\pi N_a r_e^2 m_e c^2 = 0.1535\,\textbf{MeV cm}^2\textbf{mol}^{-1}}\>,
    \label{eq:bb-constant}
\end{equation}
%
where I've chosen to mark this constant factor as $\Xi$ for easier reference in further calculations.
We can find the mean excitation energy $I$ from the following experimentally determined formula:
%
\begin{equation}
    I = \begin{cases}
        Z(12 + \frac{7}{Z})\>\rm{eV} & \text{for } Z < 13\>, \\
        Z(9.76 + 58.8Z^{-1.19})\>\rm{eV} & \text{for } Z \geq 13\>.
    \end{cases}
    \label{eq:I}
\end{equation}
The maximum energy transfer in a single collision $W_{\text{max}}$ can be calculated as:
%
\begin{equation}
    W_{\text{max}} = \frac{2m_e c^2\beta^2\gamma^2}{1 + 2\gamma \left(\sfrac{m_e}{M}\right) + \left(\sfrac{m_e}{M}\right)^2}\approx 2m_ec^2\beta^2\gamma^2\>.
    \label{eq:wmax}
\end{equation}
%
For our purposes we will ignore the density effect correction $\delta$ and the shell correction $C$.

\subsubsection{Energy Loss of Charged Kaons}
Let us calculate the energy losses for charged kaons $K^+$ and $K^-$ with a rest mass of $0.493\,\text{MeV}$ and momentum of $2.5\,\text{GeV}$ in copper which has the
following properties:
%
\begin{align*}
    & \rho = 8.92\,\text{g/cm}^3\>, \\
    & Z = 29\>, \\
    & A = 63.5\,\text{g/mol}\>.
\end{align*}
%
First let us calculate the velocity $\beta$ and the Lorentz factor $\gamma$. We know that
%
\begin{equation}
    \beta = \frac{pc}{E} = \frac{pc}{\sqrt{(pc)^2 + (Mc^2)^2}}\>,
    \label{eq:beta}
\end{equation}
%
where $M$ is the mass of the particle. Thus:
%
\begin{equation}
    \beta = \frac{2.5\>\dfrac{\rm{GeV}}{c}\>c}{\sqrt{\left(2.5\>\dfrac{\rm{GeV}}{c}\>c\right)^2 + {\left(0.493\>\dfrac{\rm{MeV}}{c^2}\>c^2\right)}^2}} \approx 0.981031\>.
\end{equation}
%
\textbf{Remember to take at least 4 significant digits for the velocity $\beta$!} This is due to the logarithm in the Bethe-Bloch equation.
The Lorentz factor $\gamma$ is then:
%
\begin{equation}
    \gamma = \frac{1}{\sqrt{1-\beta^2}} \approx 5.159\>.
\end{equation}
%
Next let us calculate the maximum energy transfer in a single collision $W_{\text{max}}$:
%
\begin{equation}
    W_{\text{max}} = 2m_ec^2\beta^2\gamma^2 = 2\cdot 0.511\>\frac{\rm{MeV}}{c^2}\>c^2\cdot(0.981031)^2(5.159)^2 = 26.7\>\text{MeV}\>.
\end{equation}
%
Last prerequisite is the mean excitation energy $I$ which we can calculate using the formula (\ref{eq:I}) for $Z \geq 13$:
%
\begin{equation}
    I = 29(9.76 + 58.8\cdot 29^{-1.19})\>\rm{eV} = 313.9\>\rm{eV}\>.
\end{equation}
%
Now all that is left is to plug in the values into the Bethe-Bloch equation (\ref{eq:bethe-bloch}):
%
\begin{flalign}
    -\left\langle \frac{\dd E}{\dd x} \right\rangle &= \Xi\>\rho \frac{Z}{A}\frac{z^2}{\beta^2}\left[\ln\left(\frac{W_{\text{max}}^2}{I^2}\right) - 2\beta^2\right] \nonumber \\
    &= 0.1535\>\frac{\rm{MeV}\,\rm{cm}^2}{\rm{mol}}\cdot 8.92\>\frac{\rm{g}}{{\rm{cm}}^3}\cdot\frac{29}{63.5}\>\frac{\rm{mol}}{\rm{g}}\cdot\frac{1}{0.981031^2}\cdot\left[\ln\left(\frac{(26.7\cdot10^6\>\rm{eV})^2}{(313.9\>\rm{eV})^2}\right) - 2\cdot (0.981031)^2\right] \nonumber \\
    &= 13.47\>\frac{\rm{MeV}}{\rm{cm}} \>.
    \label{eq:1sub1-result}
\end{flalign}
%
Thus the energy loss of charged kaons $K^+$ and $K^-$ with a momentum of $2.5\,\rm{GeV}$ in copper is $13.47\,\rm{MeV/cm}$.

\subsubsection{What is the Energy Resolution of the Detector from the Previous Example?}
Let's calculate the energy resolution of the detector from the previous example, assuming that the length of the 
particle track through the detector is $d = 5\,\rm{cm}$ and that energy is measured based on all deposited energy without
any additional losses. Using the result from the previous example (\ref{eq:1sub1-result}), we can calculate the average energy deposited 
in the detector as:
%
\begin{equation}
    \Delta E = \bar{\Delta} = -\left\langle \frac{\dd E}{\dd x} \right\rangle \cdot d = 13.47\>\frac{\rm{MeV}}{\rm{cm}}\cdot 5\>\rm{cm} = 67.35\>\rm{MeV}\>.
\end{equation}
%
This is an approximation since we are assuming that $\beta$ is constant throughout the detector, which is not true. In reality we'd 
have to integrate the energy loss over the path of the particle, however at $p\sim\rm{GeV}$ additional losses of $\sim\rm{MeV}$ are 
negligible. Measurements of energy are dependant on the energy resolution $R$ which is defined as:
%
\begin{equation}
    R = \frac{\sigma_E}{\bar{\Delta}}\>,
    \label{eq:energy-res}
\end{equation}
%
where $\sigma_E$ is the standard deviation of the energy measurement which we assume to have a Gaussian distribution like such:
%
\begin{equation}
    p(\Delta) = \frac{1}{\sqrt{2\pi}\sigma_E}\exp\left(-\frac{{\Delta - \bar{\Delta}}^2}{2\sigma_E^2}\right)\>.
\end{equation}
%
$\sigma_E$ is determined empirically. For \textbf{non-relativistic} particles it can be calculated as the variance of the 
Bethe-Bloch equation as:
%
\begin{equation}
    \sigma_0^2 = 4\pi N_a r_e^2 (m_e c^2)^2 \rho \frac{Z}{A}\,\Delta x\>.
    \label{eq:sigE-nonrel}
\end{equation}
%
For \textbf{relativistic} particles we can correct the variance from (\ref{eq:sigE-nonrel}) as such:
%
\begin{equation}
    \sigma_E^2 = \sigma_0^2\>\frac{1-\frac{1}{2}\beta^2}{1-\beta^2}\>.
\end{equation}
%
In our case this gives us:
%
\begin{flalign}
    \sigma_E^2 &= 2\cdot 0.511\>\frac{\rm{MeV}}{c^2}\>c^2\cdot 0.1535\>\frac{\rm{MeV}\rm{cm}^2}{\rm{mol}}\cdot 8.92\>\frac{\rm{g}}{{\rm{cm}}^3}\cdot\frac{29}{63.5}\>\frac{\rm{mol}}{\rm{g}}\cdot 5\>\rm{cm}\cdot\frac{1-\frac{1}{2}(0.981031)^2}{1-(0.981031)^2} \nonumber \\
    &= 44.12\>\rm{MeV}^2\>.
\end{flalign}
%
Thus the energy resolution of the detector is:
%
\begin{equation}
    R = \frac{\sqrt{44.12\>\rm{MeV}^2}}{67.35\>\rm{MeV}} = 9.9\%\>.
    \label{eq:1sub2-result}
\end{equation}
%
\subsubsection{What if the Detector is Made of a Molecule?}
Let's assume now that our detector is made of lead(II) fluoride $\text{PbF}_2$ in a cubic crystal form which has the following properties:
%
\begin{equation*}
    \begin{aligned}
        & \rho = 7.77\>\text{g/cm}^3\>, \\
        & Z = 100\>, \\
        & A = 245.2\>\text{g/mol}\>. \\
        & A_{\rm{Pb}} = 207.2\>\text{u}\>, \\
        & A_{\rm{F} }= 19\>\text{u}\>, \\
    \end{aligned}
    \qquad\qquad
    \begin{aligned}
        & Z_{\rm{Pb}} = 82\>, \\
        & Z_{\rm{F} }= 9\>, \\
        & \rho_{\rm{Pb}} = 11.34\>\text{g/cm}^3\>, \\
        & \rho_{\rm{F}} = 0.001696\>\text{g/cm}^3\>.
    \end{aligned}
\end{equation*}
%
We are interested in the energy loss of protons with a momentum of $3\,\text{GeV}$ in such a detector. The difference between calculating the energy loss in a compound 
material is that we have to calculate the energy loss for each element in the compound. This sum is weighted by the fraction of the element in the compound. As such:
%
\begin{equation}
    \frac{1}{\rho} \left\langle \frac{\dd E}{\dd x} \right\rangle _{\text{compound}} = \frac{w_1}{\rho_1} \left\langle \frac{\dd E}{\dd x} \right\rangle _1 + \frac{w_2}{\rho_2} \left\langle \frac{\dd E}{\dd x} \right\rangle _2 + \ldots\>,
    \label{eq:bb-compound}
\end{equation}
%
where we calculate $w_i$ as:
%
\begin{equation}
    w_i = \frac{a_i\cdot A_i}{\sum a_i\cdot A_i}\>,
\end{equation}
%
here $a_i$ is the number of atoms of the element in the compound and $A_i$ is the atomic mass of the element. Our professor stated that such 
problems will not be present on the exam and that we should not worry about them. However it is still good to know how to calculate the energy loss in a compound.
In our case we can expect to get effective values if the detector is made of a compound. If we calculate the weights for lead and fluorine in lead(II) fluoride we get:
%
\begin{align*}
    w_{\rm{Pb}} = \frac{1\cdot 207.2\>\rm{u}}{1\cdot 207.2\>\rm{u} + 2\cdot 19\>\rm{u}} = 0.845 \>, \\
    w_{\rm{F}} = \frac{2\cdot 19\>\rm{u}}{1\cdot 207.2\>\rm{u} + 2\cdot 19\>\rm{u}} = 0.154 \>,
\end{align*}
%
where we used the atomic masses of lead and fluoride in atomic mass units. Next we need to calculate the velocity $\beta$ and the Lorentz factor $\gamma$ for protons. So using
(\ref{eq:beta}) we get:
%
\begin{equation}
    \beta = \frac{3\>\dfrac{\rm{GeV}}{c}\>c}{\sqrt{\left(3\>\dfrac{\rm{GeV}}{c}\>c\right)^2 + {\left(0.938\>\dfrac{\rm{GeV}}{c^2}\>c^2\right)}^2}} \approx 0.95443\>,
\end{equation}
%
which gives us a Lorentz factor of:
%
\begin{equation}
    \gamma = \frac{1}{\sqrt{1-(0.95443)^2}} \approx 3.351\>.
\end{equation}
%
Next we need to calculate the maximum energy transfer in a single collision $W_{\text{max}}$ using 
(\ref{eq:wmax}) as: 
%
\begin{equation}
    W_{\text{max}} = 2\cdot 0.511\>\frac{\rm{MeV}}{c^2}\>c^2\cdot(0.95443)^2(3.351)^2 = 10.5\>\rm{MeV}\>,
\end{equation}
%
and the mean excitation energy $I$ using (\ref{eq:I}) for each component:
%
\begin{align*}
    I_{\rm{Pb}} &= 82\left(9.76 + 58.8\cdot 82^{-1.19}\right)\rm{eV} = 825.8\>\rm{eV}\>, \\
    I_{\rm{Cu}} &= 9\left(12 + \frac{7}{9}\right)\rm{eV} = 115\>\rm{eV}\>.
\end{align*}
%
Now we can calculate the energy loss for each component using the Bethe-Bloch equation (\ref{eq:bethe-bloch}) and sum them up:
%
\begin{flalign}
    -\left\langle \frac{\dd E}{\dd x} \right\rangle _{\rm{Pb}} &= \Xi\>\rho_{\rm{Pb}} \frac{Z_{\rm{Pb}}}{A_{\rm{Pb}}}\frac{1}{0.95443^2}\left[\ln\left(\frac{(10.5\cdot10^6\>\rm{eV})^2}{(825.8\>\rm{eV})^2}\right) - 2\cdot (0.95443)^2\right] \nonumber \\
    &= \Xi\>\rho_{\rm{Pb}} \frac{Z_{\rm{Pb}}}{A_{\rm{Pb}}} \cdot 18.7488 \nonumber \\
    &= 0.1535\>\frac{\rm{MeV}\,\rm{cm}^2}{\rm{mol}}\cdot 11.34\>\frac{\rm{g}}{{\rm{cm}}^3}\cdot\frac{82}{207.2}\>\frac{\rm{mol}}{\rm{g}}\cdot 18.7488 \nonumber \\
    &= 12.9\>\frac{\rm{MeV}}{\rm{cm}}\>, \\
    -\left\langle \frac{\dd E}{\dd x} \right\rangle _{\rm{F}} &= \Xi\>\rho_{\rm{F}} \frac{Z_{\rm{F}}}{A_{\rm{F}}}\frac{1}{0.95443^2}\left[\ln\left(\frac{(10.5\cdot10^6\>\rm{eV})^2}{(115\>\rm{eV})^2}\right) - 2\cdot (0.95443)^2\right] \nonumber \\
    &= \Xi\>\rho_{\rm{F}} \frac{Z_{\rm{F}}}{A_{\rm{F}}} \cdot 23.0773 \nonumber \\
    &= 0.1535\>\frac{\rm{MeV}\,\rm{cm}^2}{\rm{mol}}\cdot 0.001696\>\frac{\rm{g}}{{\rm{cm}}^3}\cdot\frac{9}{19}\>\frac{\rm{mol}}{\rm{g}}\cdot 23.0773 \nonumber \\
    &= 0.002846\>\frac{\rm{MeV}}{\rm{cm}}\>.
\end{flalign}
%
Now all that is left is to compute the weighted sum as stated in (\ref{eq:bb-compound}):
%
\begin{flalign}
    -\left\langle \frac{\dd E}{\dd x} \right\rangle _{\text{compound}} &= -\frac{\rho\cdot w_{\rm{Pb}}}{\rho_{\rm{Pb}}}\cdot\left\langle \frac{\dd E}{\dd x} \right\rangle _{\rm{Pb}} - \frac{\rho\cdot w_{\rm{F}}}{\rho_{\rm{F}}}\cdot\left\langle \frac{\dd E}{\dd x} \right\rangle _{\rm{F}} \nonumber \\
    &= \frac{7.77\>\dfrac{\rm{g}}{\rm{cm}^3}\cdot 0.845}{11.34\>\dfrac{\rm{g}}{\rm{cm}^3}}\cdot 12.9\>\frac{\rm{MeV}}{\rm{cm}} + \frac{7.77\>\dfrac{\rm{g}}{\rm{cm}^3}\cdot 0.154}{0.001696\>\dfrac{\rm{g}}{\rm{cm}^3}}\cdot 0.002846\>\frac{\rm{MeV}}{\rm{cm}} \nonumber \\
    &= 9.47 \>\frac{\rm{MeV}}{\rm{cm}}\>.
    \label{eq:1sub3-result}
\end{flalign}
%


% Add references
% \newpage
% \bibliographystyle{unsrt}
% \bibliography{main}

% End document
\end{document}
